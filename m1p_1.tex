\documentclass[12pt, twoside]{article}
\usepackage{jmlda}
\newcommand{\hdir}{.}

\begin{document}

\title
    [Обоснованность применения DTW к пространственно-временным объектам] % краткое название; не нужно, если полное название влезает в~колонтитул
    {Теоретическая обоснованность применения метрических методов классификации с использованием динамического выравнивания (DTW) к пространственно-временным объектам}
\author
    [И.\,О.~Автор] % список авторов (не более трех) для колонтитула; не нужен, если основной список влезает в колонтитул
    {И.\,О.~Автор, И.\,О.~Соавтор, И.\,О.~Фамилия} % основной список авторов, выводимый в оглавление
    [И.\,О.~Автор$^1$, И.\,О.~Соавтор$^2$, И.\,О.~Фамилия$^{1,2}$] % список авторов, выводимый в заголовок; не нужен, если он не отличается от основного
\email
    {author@site.ru; co-author@site.ru;  co-author@site.ru}
\thanks
    {Работа выполнена при
     %частичной
     финансовой поддержке РФФИ, проекты \No\ \No 00-00-00000 и 00-00-00001.}
\organization
    {$^1$Организация, адрес; $^2$Организация, адрес}
\abstract
    {В работе исследуется корректность применения методов динамического выравнивания (DTW) и его модификаций к пространственно-временным рядам. При доказательстве, проверяют, что функция, порождаемая алгоритмом динамического выравнивания (DTW), является ядром, что обосновывает применение метрических методов классификации. Проверка того, является ли функция ядром осуществляется при помощи теоремы Мерсера, основная часть которой заключается в проверке матрицы попарных расстояний на неотрицательную определенность. 
	
\bigskip
\noindent
\textbf{Ключевые слова}: \emph {алгоритм динамического выравнивания (DTW); пространственно-временные ряды; ядро функции; теорема Мерсера; метод опорных векторов (SVM)}
}


%данные поля заполняются редакцией журнала
\doi{10.21469/22233792}
\receivedRus{01.01.2017}
\receivedEng{January 01, 2017}

\maketitle
\linenumbers

\section{Введение}
{Функция расстояния между временными рядами может быть задана различными
способами: Евклидово расстояние \cite{Evklid}, метод динамического выравнивания временных
рядов \cite{DTW1}, \cite{DTW2}, метод, основанный на нахождение наибольшей общей последовательности \cite{MaxSequence}.
В \cite{Noise} показано, что разность между значениями временного ряда, соответствующими различным временным отсчетам, не может рассматриваться в качестве описания расстояния между двумя объектами: эта мера расстояния чувствительна к
шуму и локальным временным сдвигам. Для решения задачи предлагается использовать метод динамического выравнивания временных рядов (англ. Dynamic Time
Warping) \cite{DTW}. Как показано в \cite{Thebestmethod}, этот метод находит наилучшее соответствие между
двумя временными рядами, если они нелинейно деформированы друг относительно
друга -- растянуты, сжаты или смещены вдоль оси времени. Метод DTW используется для определения сходства между ними и введения расстояния между двумя
объектами.\\

На данный момент существуют теоретические обоснования применения DTW лишь для некоторых временных объектов, например, для дизартрического распознавания речи с разреженными обучающими данными \cite{DisartreSpeech}. В этой статье мы теоретически обоснуем его применение для пространственно-временных объектов. В первую очередь, это будет происходить на данных измерения активности мозга обезьян \cite{Data}.\\


Алгоритм построения оптимальной разделяющей гиперплоскости -- алгоритм линейной классификации \cite{SVM_Bennett}. В основе создания же нелинейного классификатора лежит замена скалярного произведения $\langle x, x' \rangle$ на функцию ядра $K(x, x')$. Таким образом осуществляется переход в спрямляющее пространство (kernel trick), который позволяет построить нелинейные разделители. Если изначально выборка была линейно неразделимой, то при удачном выборке ядра можно избавить от этой проблемы. Это позволяет применять линейные алгоритмы классификации (SVM) в случаях, когда выборка не разделяется линейно. Критерием функции ядра является теорема Мерсера \cite{Mercer},\\




}





\section{Постановка задачи}
В работе мы будем проверять выполнение условий теоремы Мерсера на разных данных для разных модификаций DTW. То есть, следующие два условия для функции $K(x,x')$, порожденной DTW:\\
$\bullet \ K(x,x') = K(x',x)$\\
$\bullet \ \int\limits_X \int\limits_X K(x,x')g(x)g(x')dxdx' \geq 0 \ \ \ \forall g:X \rightarrow \mathbb{R}$\\
Последнее условие эквивалентно тому, что для любых наборов $\{ x_1, \ldots, x_n \}$ матрица $K = ||K(x_i,x_j)||_{i,j}$ неотрицательно определена: $v^TKv \geq 0 \ \ \forall v \in \mathbb{R}^n$





\section{Заключение}
Желательно, чтобы этот раздел был, причём он не~должен дословно повторять аннотацию.
Обычно здесь отмечают, каких результатов удалось добиться, какие проблемы остались открытыми.

%%%% если имеется doi цитируемого источника, необходимо его указать, см. пример в \bibitem{article}
%%%% DOI публикации, зарегистрированной в системе Crossref, можно получить по адресу http://www.crossref.org/guestquery/

%%%% если имеется doi цитируемого источника, необходимо его указать, см. пример в \bibitem{article}
%%%% DOI публикации, зарегистрированной в системе Crossref, можно получить по адресу http://www.crossref.org/guestquery/.

\bibliographystyle{plain}
\bibliography{litr}

\end{document}
